%Especificar todo lo que aplique de manera general en el diseño
%Lo que se requiere para comprar el diseño
%Acuerdos y estandares 

%(Variables significativas, mismos frames, mismos metodos, etc.)

%--------------------------------------------------
\section{Especificaciones del Diseño}

%%%%%%%%%%%%%%%


%--------------------------------------------------
\section{Compra del Diseño}

A continuaci\'on se .....

%--------------------------------------------------
\section{Acuerdos y Estandares}

Est\'andares para el mantenimiento del software es un an\'alisis automatizado de las aplicaciones es decisivo para reducir costes. Los est\'andares para los procesos del ciclo de vida del software permiten conectar y asociar el proceso de mantenimiento con los dem\'as procesos existentes en el software.

Existen diversos est\'andares que tienen una relaci\'on directa o indirecta con el mantenimiento del software:
\begin{itemize}
\item Para los procesos del ciclo de vida del software: IEEE 1074 e ISO 12207.
\item Para la calidad del software y sus m\'etricas: IEEE 1061 e ISO 9126.

\item Para el mantenimiento del software: 
\begin{itemize}
\item IEEE 1219: Describe un proceso iterativo para la gesti\'on y ejecuci\'on de las actividades del proceso. Aunque s\'olo menciona las fases de desarrollo y de producci\'on de un producto de software, \'estas cubren todo su ciclo de vida, cualquiera que sea su tama\~no o complejidad.
 
\item ISO/IEC 14764: Proceso de mantenimiento del software, contiene las actividades y tareas del mantenedor, proporciona una gu\'ia que explica c\'omo llevar a cabo el proceso de mantenimiento y establece definiciones para los distintos tipos de mantenimiento. La gu\'ia es aplicable a la planificaci\'on, ejecuci\'on y control, mantenimiento, revisi\'on y evaluaci\'on del proceso de mantenimiento.
\end{itemize}
\end{itemize}




