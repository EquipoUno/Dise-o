\subsection{IURegistroCita Pantalla de registro de cita}

\subsection{Objetivo}
	Permitir al cliente que seleccione la cita en d\'ia y hora deseada.

\subsection{Diseño}
	Esta pantalla muestra la opci\'on Fecha y la opci\'on Hora. Aqu\'i cliente puede seleccionar la hora y la fecha de la cita m\'edica.

\IUfig[.9]{gui/IURegistroCita}{IURegistroCita}{Pantalla de Registro de Cita.}

\subsection{Salidas}

	Pantalla de confirmacion de cupo.

\subsection{Entradas}
Fecha y Hora.

\subsection{Comandos}
\begin{itemize}
		\item \IUbutton{Fecha}: Muestra un calendario, para que el actor pueda seleccionar la fecha que mas le convenga.
		\item \IUbutton{Hora}: Muestra los horarios de las citas en intervalos de 30 minutos.
		\item \IUbutton{Revisar cupo}: Muestra una nueva pantalla en la cual se muestran dos pantallas diferentes, la primera en el caso de que haya cupo en los consultorios y la segunda en caso de que no haya cupo.

\IUfig[.9]{gui/IUCupo}{IUCupo}{Pantalla de cupo en los consultorios.}
\IUfig[.9]{gui/IUNoCupo}{IUNoCupo}{Pantalla de no cupo en los consultorios.}

		\item \IUbutton{Confirma cita}: Nos permite generar el recibo para que la cita se guarde en el sistema.
		\item \IUbutton{Escoger otro horario}: En caso de que haya cupo o no en los consultorios, el actor puede rectificar su cita.
\end{itemize}

\subsection{Mensajes}
	\begin{Citemize}
		\item {\bf ERRNoCupo} - "No hay cupo en ning\'un consultorio para ese horario. Porfavor seleccone un horario o fecha distintos".
	\end{Citemize}

